% LaTeX Curriculum Vitae Template
%
% Copyright (C) 2004-2009 Jason Blevins <jrblevin@sdf.lonestar.org>
% http://jblevins.org/projects/cv-template/
%
% You may use use this document as a template to create your own CV
% and you may redistribute the source code freely. No attribution is
% required in any resulting documents. I do ask that you please leave
% this notice and the above URL in the source code if you choose to
% redistribute this file.

\documentclass[letterpaper]{article}

\usepackage[dvipdfmx]{graphicx}
\usepackage{fancybox}
\usepackage{longtable}
\usepackage{fancyvrb}
\usepackage[dvipdfmx]{hyperref}
\usepackage{url}
\usepackage[dvipdfmx]{color}
\usepackage{nextpage}
\usepackage{float}
\usepackage[table,dvipdfmx]{xcolor}
\usepackage{wrapfig}
\usepackage{tabularx}
\usepackage{multicol}
\usepackage{makeidx}
\usepackage{ascmac} % screen
\usepackage[dvips]{xy} % for advi workaround. Bug #452044

\usepackage{geometry}

% Comment the following lines to use the default Computer Modern font
% instead of the Palatino font provided by the mathpazo package.
% Remove the 'osf' bit if you don't like the old style figures.
%\usepackage[T1]{fontenc}
%\usepackage[sc,osf]{mathpazo}

% Set your name here
\def\name{Kiwamu Okabe - Research and Development Engineer}

% Replace this with a link to your CV if you like, or set it empty
% (as in \def\footerlink{}) to remove the link in the footer:
\def\footer{
  \begin{center}
    \begin{footnotesize}
      Last updated: \today
    \end{footnotesize}
  \end{center}
}

% The following metadata will show up in the PDF properties
\hypersetup{
  colorlinks = true,
  urlcolor = black,
  pdfauthor = {\name},
  pdfkeywords = {economics, statistics, mathematics},
  pdftitle = {\name: Curriculum Vitae},
  pdfsubject = {Curriculum Vitae},
  pdfpagemode = UseNone
}

\geometry{
  body={6.5in, 8.5in},
  left=1.0in,
  top=1.25in
}

% Customize page headers
\pagestyle{myheadings}
\markright{\name}
\thispagestyle{empty}

% Custom section fonts
\usepackage{sectsty}
\sectionfont{\rmfamily\mdseries\Large}
\subsectionfont{\rmfamily\mdseries\itshape\large}

% Other possible font commands include:
% \ttfamily for teletype,
% \sffamily for sans serif,
% \bfseries for bold,
% \scshape for small caps,
% \normalsize, \large, \Large, \LARGE sizes.

% Don't indent paragraphs.
\setlength\parindent{0.2em}

%% Make lists without bullets
%\renewenvironment{itemize}{
%  \begin{list}{}{
%    \setlength{\leftmargin}{1.5em}
%  }
%}{
%  \end{list}
%}

\begin{document}

% Place name at left
{\huge \name}

% Alternatively, print name centered and bold:
%\centerline{\huge \bf \name}

\vspace{0.25in}

\begin{minipage}{0.3\linewidth}
  \begin{tabular}{ll}
    Phone: & +81-90-3524-7064 \\
    Email: & \href{mailto:kiwamu@debian.or.jp}{\tt kiwamu@debian.or.jp} \\
    Homepage: & \href{http://masterq.metasepi-design.com/}{\tt http://masterq.metasepi-design.com/} \\
  \end{tabular}
\end{minipage}

\section*{Interests}

I launched my career on developing embedded devices at Ricoh Company, Ltd, and also learned application design using functional language such like Haskell. I found Metasepi project\footnote{\href{http://www.metasepi.org/}{\tt http://www.metasepi.org/}} what is trying to apply strong type to embedded programming. For my first challenging, I developed an embedded Haskell compiler named Ajhc\footnote{\href{http://ajhc.metasepi.org/}{\tt http://ajhc.metasepi.org/}}, and published some research papers\footnote{\href{http://www.metasepi.org/papers.html}{\tt http://www.metasepi.org/papers.html}}.

Today, I choose ATS language\footnote{\href{http://www.ats-lang.org/}{\tt http://www.ats-lang.org/}} as embedded functional language, and found ``Japan ATS User Group\footnote{\href{http://jats-ug.metasepi.org/}{\tt http://jats-ug.metasepi.org/}}'' what holds Japanese translations about the language. And I'm implementing\footnote{\href{http://fpiot.metasepi.org/}{\tt http://fpiot.metasepi.org/}} these technology on tiny MCU such like ARM Cortex-M series and 8bit AVR for practical use.

Metasepi is very experimental and ambitious project, however I believe that it also introduces an by-product ``the technology to design real software with predictable manpower and safety'' regardless of embedded domain.

\section*{Work Experience}
%%%%%%%%%%%%%%%%
\subsection*{November 2016 - Present: Expert Engineer (permanent employee) at SELTECH CORPORATION}

\begin{itemize}
  \item Maintain a Hypervisor for embedded market
  \item Design and develop own secure OS for ARM platform
\end{itemize}

%%%%%%%%%%%%%%%%
\subsection*{August 2014 - Present: Part-time Researcher at RIKEN Advanced Institute for Computational Science}

\begin{itemize}
  \item Research embedded functional programming running on ARM Cortex-M and AVR
  \item Verification for RTOS application such like ChibiOS/RT\footnote{\href{http://www.chibios.org/}{\tt http://www.chibios.org/}} running on ARM Cortex-M
\end{itemize}

%%%%%%%%%%%%%%%%
\subsection*{July 2013 - Present: Self-employed Software Engineer at METASEPI DESIGN}

\begin{itemize}
  \item Research and develop Ajhc Haskell Compiler
  \item Host meetups\footnote{\href{https://metasepi.connpass.com/}{\tt https://metasepi.connpass.com/}} for hands-on to verify embedded application on ARM Cortex-M using STM32\footnote{\href{http://www.st.com/en/microcontrollers/stm32-32-bit-arm-cortex-mcus.html}{\tt http://www.st.com/en/microcontrollers/stm32-32-bit-arm-cortex-mcus.html}} board and ST-LINK\footnote{\href{http://www.st.com/en/development-tools/st-link.html}{\tt http://www.st.com/en/development-tools/st-link.html}} debugger
  \item ATS language evangelist
  \item Verification evangelist using VeriFast\footnote{\href{https://github.com/verifast/verifast}{\tt https://github.com/verifast/verifast}}, which is a verifier C language programs annotated with preconditions and postconditions
  \item Translated VeriFast Tutorial into Japanese\footnote{\href{https://github.com/jverifast-ug/translate/blob/master/Manual/Tutorial/Tutorial.md}{\tt https://github.com/jverifast-ug/translate/blob/master/Manual/Tutorial/Tutorial.md}}
  \item Support to develop any embedded software
  \item Manage Metasepi Project and develop the core technology
\end{itemize}

%%%%%%%%%%%%%%%%
\subsection*{February 2016 - November 2016: Software Enginner (contract employee) at Life Robotics Inc.}

\begin{itemize}
  \item Design GUI application running on Linux OS, using C++ and Qt\footnote{\href{https://www.qt.io/}{\tt https://www.qt.io/}}
  \item Design network protocol for Robotics application
  \item Reason for Quitting: to challenge verification of secure application using C language and VeriFast
\end{itemize}

%%%%%%%%%%%%%%%%
\subsection*{March 2015 - February 2016: System Enginner (contract employee) at Centillion Japan Co., Ltd.}

\begin{itemize}
  \item Technical support for stock chart application using JavaScript
  \item Maintain MySQL database servers
  \item Launch new IoT business for farming
  \item Design a platform\footnote{\href{https://github.com/centillion-tech/kick-r}{\tt https://github.com/centillion-tech/kick-r}} to accelerate R programs
  \item Reason for Quitting: to change my job into embedded one, again
\end{itemize}

%%%%%%%%%%%%%%%%
\subsection*{September 2014 - December 2014: Software engineer (trustee agreement) at Axsh co., LTD.}

\begin{itemize}
  \item Develop an OpenFlow application named ``OpenVNet''\footnote{\href{https://github.com/axsh/openvnet}{\tt https://github.com/axsh/openvnet}}
  \item Design automation scripts for AWS using Ruby and GNU make
  \item Reason for Quitting: to have better salary
\end{itemize}

%%%%%%%%%%%%%%%%
\subsection*{March 2012 - July 2013: Software Engineer (permanent employee) at MIRACLE LINUX CORPORATION}

\begin{itemize}
  \item Maintain own Digital Signage platform running on Intel architecture using Linux OS, C++, OpenGL, GTK+\footnote{\href{https://www.gtk.org/}{\tt https://www.gtk.org/}} and GStreamer\footnote{\href{https://gstreamer.freedesktop.org/}{\tt https://gstreamer.freedesktop.org/}}
  \item Verify and tune up performance of Digital Signage on new Intel platform and Intel video driver
  \item Verify PowerPC Linux kernel and debug/fix a race condition in the SMP kernel
  \item Debug and fix bug of crash\footnote{\href{http://people.redhat.com/{\textasciitilde}anderson/}{\tt http://people.redhat.com/{\textasciitilde}anderson/}} command's PowerPC virtual memory paging
  \item Design new Windows installer using NSIS\footnote{\href{http://nsis.sourceforge.net/}{\tt http://nsis.sourceforge.net/}}
  \item Introduce and maintain new Git server for internal use
  \item Reason for Quitting: to focus researching and developing Ajhc Haskell compiler on full-time work
\end{itemize}

%%%%%%%%%%%%%%%%
\subsection*{April 2001 - February 2012: Software Development Engineer (permanent employee) at Ricoh Company, Ltd.}

\begin{itemize}
  \item Develop BIOS and bootloader for multifunction printer on Intel architecture
  \item Design secure boot for multifunction printer on Intel architecture
  \item Develop new BIOS for multifunction printer
  \item Tune multifunction printer boot time as 10 seconds
  \item Develop POSIX thread library
  \item Develop and technical support NetBSD OS
  \item Port OS to new Intel hardware
  \item Reason for Quitting: to join more small and quick team
\end{itemize}

\section*{Education}

\begin{itemize}
  \item March 2001: Master of Engineering from Department of Electrical and Electronic Engineering, Tokyo Metropolitan University. \\
    The thesis: ``Multimode Quartz Crystal Microbalance''\footnote{\href{http://ci.nii.ac.jp/naid/110004076869}{\tt http://ci.nii.ac.jp/naid/110004076869}}
\end{itemize}

\section*{Publications and Reports}

\begin{itemize}
  \item Kiwamu Okabe and Hongwei Xi. ``Arduino programing of ML-style in ATS''\footnote{\href{http://www.metasepi.org/doc/metasepi-icfp2015-arduino-ats.pdf}{\tt http://www.metasepi.org/doc/metasepi-icfp2015-arduino-ats.pdf}}. ML workshop, 2015.
  \item Kiwamu Okabe and Takayuki Muranushi. ``Systems Demonstration: Writing NetBSD Sound Drivers in Haskell''\footnote{\href{http://metasepi.org/doc/metasepi-icfp2014-demo.pdf}{\tt http://metasepi.org/doc/metasepi-icfp2014-demo.pdf}}. Haskell Symposium, 2014.
  \item Kiwamu Okabe. ``ATS言語を使って不変条件をAPIに強制する''.\footnote{\href{http://www.metasepi.org/doc/20141101\_prosym\_summer2014.pdf}{\tt http://www.metasepi.org/doc/20141101\_prosym\_summer2014.pdf}} 夏のプログラミング・シンポジウム 2014, 2014.
  \item Kiwamu Okabe, Hiroki MIZUNO and Hidekazu SEGAWA. ``強い型によるOSの開発手法の提案''\footnote{\href{http://metasepi.org/doc/20140110\_prosym55.pdf}{\tt http://metasepi.org/doc/20140110\_prosym55.pdf}}. 第55回プログラミング・シンポジウム, 2014.
\end{itemize}

\section*{Activities}

\subsection*{Open-source projects}

\subsubsection*{Metasepi Project\footnote{\href{http://metasepi.org/}{\tt http://metasepi.org/}}}
\begin{itemize}
\item Challenge to create an open-source Unix-like operating system designed with strong type such as ML or Haskell.
\item Rewriting NetBSD kernel using Ajhc Haskell compiler. \href{https://github.com/metasepi/netbsd-arafura-s1}{\tt https://github.com/metasepi/netbsd-arafura-s1}
\end{itemize}

\subsubsection*{Ajhc Haskell compiler\footnote{\href{http://ajhc.metasepi.org/}{\tt http://ajhc.metasepi.org/}}}
\begin{itemize}
\item Extend and add embedded features to Jhc Haskell Compiler \href{http://repetae.net/computer/jhc/}{\tt http://repetae.net/computer/jhc/}.
\item Ajhc has thread-safe and reentrant runtime. Also has Erlang style GC. It means Ajhc's Haskell context has own GC heap. GC can run on tiny CPU such as Cortex-M3 with 32kB RAM.
\end{itemize}

\subsubsection*{Japan ATS User Group\footnote{\href{http://jats-ug.metasepi.org/}{\tt http://jats-ug.metasepi.org/}}}
\begin{itemize}
\item An user group for ATS language promotion of utilization. Translating ATS documents into Japanese.
\end{itemize}

\subsubsection*{Debian Maintainer\footnote{\href{http://qa.debian.org/developer.php?login=kiwamu@debian.or.jp}{\tt http://qa.debian.org/developer.php?login=kiwamu@debian.or.jp}}}
\begin{itemize}
\item Maintained uim package at Debian squeeze, and packages using Haskell at sid.
\end{itemize}

\subsubsection*{Carettah\footnote{\href{https://github.com/master-q/carettah}{\tt https://github.com/master-q/carettah}}}
\begin{itemize}
\item A presentation tool written with Haskell. All of my slides\footnote{\href{http://www.slideshare.net/master\_q/}{\tt http://www.slideshare.net/master\_q/}} are created by the tool.
\end{itemize}

\section*{Computer Skills}

\begin{itemize}
  \item Languages: C, C++, Haskell, ATS, Intel/ARM assembler, Ruby, JavaScript, Python
  \item Platforms: Linux, NetBSD, FreeRTOS, ChibiOS/RT, Android NDK, Cygwin, MinGW, Bare metal
\end{itemize}

\section*{Reference available upon request}

\begin{itemize}
  \item Hiroyasu Fukuyama CTO - SELTECH CORPORATION
  \item Woo-Keun Yoon CEO - Life Robotics Inc.
  \item Kentaro Kuroiwa Research Chief - Centillion Japan Co., Ltd.
  \item Yasuhiro Yamazaki CEO - Axsh Co., Ltd.
  \item Takayuki Muranushi - RIKEN Advanced Institute for Computational Science
  \item Takashi KODAMA CEO - MIRACLE LINUX CORPORATION
  \item Shigeya SENDA - Ricoh Company, Ltd.
  \item Hitoshi Sekimoto Professor - Tokyo Metropolitan University, Department of Electrical and Electronic Engineering
\end{itemize}

\bigskip
\footer

\end{document}
