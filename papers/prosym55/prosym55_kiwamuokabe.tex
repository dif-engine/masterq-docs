%%
%% 研究報告用スイッチ
%% [techrep]
%%
%% 欧文表記無しのスイッチ(etitle,jkeyword,eabstract,ekeywordは任意)
%% [noauthor]
%%

%\documentclass[submit,techreq]{ipsj}
\documentclass[submit,techreq,noauthor]{ipsj}


\usepackage[dvips]{graphicx}
\usepackage{latexsym}

\def\Underline{\setbox0\hbox\bgroup\let\\\endUnderline}
\def\endUnderline{\vphantom{y}\egroup\smash{\underline{\box0}}\\}
\def\|{\verb|}


\begin{document}

% Title %%%%%%%%%%%%%%%%%%%%%%%%%%%%%%%%%
\title{強い型によるOSの開発手法の提案}

\affiliate{METASEPI}{Metasepi Project}

\author{岡部 究}{Kiwamu Okabe}{METASEPI}[kiwamu@debian.or.jp]
\author{Hiroki MIZUNO}{Hiroki MIZUNO}{}
\author{瀬川 秀一}{Hidekazu SEGAWA}{}

\begin{abstract}
本稿ではMLやHaskell言語のような型推論を持つ言語を使い xxx
\end{abstract}

\begin{jkeyword}
プログラミング・シンポジウム,プログラミング言語,コンパイラ,Haskell,OS
\end{jkeyword}

\maketitle

% Body %%%%%%%%%%%%%%%%%%%%%%%%%%%%%%%%%
\section{はじめに}

\subsection{アウトライン xxx}

筆者らはjhc Haskellコンパイラに組込み向け拡張を加えて \cite{j-ikamusume5}
その成果をAjhc Haskellコンパイラとして公開している \cite{ajhc} 。

\begin{acknowledgment}
偉大なHaskellコンパイラを産み出し、
誰もが望みを捨ててしまっていたOS領域にかすかな光をもたらしたJohn Meachamに感謝する。
\end{acknowledgment}

% BibTeX %%%%%%%%%%%%%%%%%%%%%%%%%%%%%%%%%
\bibliographystyle{ipsjunsrt}
\bibliography{../bibtex/reference,../bibtex/jreference}

% Biography %%%%%%%%%%%%%%%%%%%%%%%%%%%%%%%%%
\begin{biography}
\profile{n}{岡部 究}{}
\profile{n}{Hiroki MIZUNO}{}
\profile{n}{瀬川 秀一}{}
\end{biography}

\end{document}
