\documentclass[DIV16,twocolumn,10pt]{scrreprt}
\usepackage{paralist}
\usepackage{graphicx}
\usepackage[final]{hcar}

%include polycode.fmt

\begin{document}

\begin{hcarentry}{Ajhc Haskell Compiler}
\report{Kiwamu Okabe}
\status{Experimental}
\participants{John Meacham, Hiroki MIZUNO, Hidekazu SEGAWA}% optional
\makeheader

\newcommand{\WhatsIsIt}{\subsubsection*{What is it?}}
\newcommand{\Usage}{\subsubsection*{Usage}}
\newcommand{\License}{\subsubsection*{License}}
\newcommand{\Demo}{\subsubsection*{Demonstrations}}

\WhatsIsIt

Ajhc is a Haskell compiler, and acronym for ``A fork of jhc''.

Jhc \url{http://repetae.net/computer/jhc/} converts Haskell code into pure C language code running with jhc's runtime. And the runtime is written with 3000 line (include comments) pure C code. It's a magic!

Ajhc's mission is to keep contribution to jhc in the repository. Because the upstream author of jhc, John Meacham, can't pull the contribution speedy. (I think he is too busy to do it.) We should feedback jhc any changes. And also Ajhc aims to provide Metasepi project with a method to rewrite NetBSD kernel using Haskell. The method is called Snatch-driven development.

Ajhc is, so to speak, an accelerator to develop jhc.

\WhatsNew

\noindent Runtime:

Get thread-safe and reentrant.

\vspace*{10pt}

\noindent GC:

Take Erlang style GC. It means Haskell context has own GC heap.

\vspace*{10pt}

\noindent Document:

Translate Jhc User's Manual into Japanese.
Translating into the other language is also easy, because the traslaters are using gettext.

\noindent \url{https://github.com/ajhc/ajhc/blob/arafura/po/ja.po}

xxx

\Demo

\noindent \url{http://www.youtube.com/watch?v=n6cepTfnFoo}

The touchable cube application is written with Haskell and compiled by Ajhc. The demo breaks the application when running GC using ndk-gdb debugger. You can get the demo source code at \url{https://github.com/ajhc/demo-android-ndk}. Also you can read slide about the detail at \url{http://www.slideshare.net/master\_q/20131020-osc-tokyoajhc}.

\vspace*{10pt}

\noindent \url{http://www.youtube.com/watch?v=C9JsJXWyajQ}

The demo is running code that compiled with Ajhc on Cortex-M3 board, mbed. It's a simple RSS reader for reddit.com, show the RSS titles on Text LCD panel. You can get the demo detail and source code at \url{https://github.com/ajhc/demo-cortex-m3}.

\vspace*{10pt}

\noindent \url{http://www.youtube.com/watch?v=zkSy0ZroRIs}

The demo is running Haskell code without any OS.
And also the clock exception handler is written with Haskell.

\Usage

You can install Ajhc at Hackage DB.

\begin{verbatim}
$ cabal install ajhc
$ ajhc --version
ajhc 0.8.0.9 (9c264872105597700e2ba403851cf3b236cb1646)
compiled by ghc-7.6 on a x86_64 running linux
$ echo 'main = print "hoge"' > Hoge.hs
$ ajhc Hoge.hs
$ ./hs.out
"hoge"
\end{verbatim}

Please read ``Ajhc User's Manual'' to know more detail. \url{http://ajhc.metasepi.org/manual.html}

\FuturePlans

Try to rewrite (snatch) NetBSD kernel driver with Haskell. If we have a luck, port some library like array or vector from GHC world. After that, we are going to report back about developing Ajhc.

\License

GPL2 or later.

\Contact
  \begin{itemize}
    \item Mailing list: \url{http://groups.google.com/group/metasepi}
    \item Bug tracker: \url{https://github.com/ajhc/ajhc/issues}
    \item Metasepi team: \url{https://github.com/ajhc?tab=members}
  \end{itemize}

\FurtherReading
  \begin{itemize}
    \item Ajhc - Haskell everywhere \url{http://ajhc.metasepi.org/}
    \item jhc \url{http://repetae.net/computer/jhc/}
    \item Metasepi Project \url{http://metasepi.org/}
    \item Snatch-driven-development \url{http://www.slideshare.net/master\_q/20131020-osc-tokyoajhc}
  \end{itemize}
\end{hcarentry}

\end{document}
